\documentclass[a4paper,12pt]{scrbook}
\usepackage{scrpage2}
\usepackage{graphicx}
\usepackage[bookmarks=true]{hyperref} %colorlinks=true
\usepackage[dvipsnames]{xcolor}
%\addtokomafont{title}{\color{red}} 
%\addtokomafont{title}{\color{RawSienna}}
%\usepackage{sidecap}
\widowpenalty=10000
\clubpenalty=10000
\deffootnote{0em}{1em}{\thefootnotemark\hspace*{.5em}}
%\usepackage{xunicode}
\usepackage{polyglossia}
\setmainlanguage{brazil}
\setotherlanguages{german,french,english}
\usepackage{fontspec}
\defaultfontfeatures{Ligatures=TeX}
\setmainfont{New Caledonia LT Std}
\setsansfont{New Caledonia LT Std}
%\setsansfont{Myriad Pro}
\setmonofont[Scale=MatchLowercase]{Consolas}
\setlength{\parskip}{0pt}
%\setheadsepline{.4pt}
%\addtokomafont{caption}{\footnotesize \bfseries}
\usepackage{pdflscape}
\usepackage{pdfpages}
\usepackage[bottom=2cm,top=3cm,left=3cm,right=2cm]{geometry}
%\usepackage[tocgraduated]{tocstyle} %isso pode ser modificado para não indentar/graduar entradas de seção (fica feio, mas se for exigido...)
%\usetocstyle{nopagecolumn}
\usepackage{lipsum}
%\setheadsepline{.4pt}[\color{BurntOrange}]
\setlength{\parskip}{0pt}
\usepackage{caption}
\captionsetup{figurewithout=chapter,labelsep=endash} %para números de figura correntes (sem capítulos); labelsep=endash: para “Figura 1 — etc” em vez de “Figura 1: etc”. 
\usepackage[titles]{tocloft} % para colocar nome “Figuras” antes do número na “Lista de Figuras”. Mais uma regra da ABNT, não siga se não precisar.
\usepackage{chngcntr}
\usepackage{setspace}


%\usepackage[titles]{tocloft}

\newlength{\mylen}

\renewcommand{\cftfigpresnum}{\figurename\enspace}
\renewcommand{\cftfigaftersnum}{ -- }
\settowidth{\mylen}{\cftfigpresnum} %pondo \cftfigaftersnum aqui dentro dá um espaço maior depois do travessão
\addtolength{\cftfignumwidth}{\mylen}

\renewcommand{\cfttabpresnum}{\tablename\enspace}
\renewcommand{\cfttabaftersnum}{ --\hspace{\enspace}}
\settowidth{\mylen}{\cfttabpresnum\cfttabaftersnum}
\addtolength{\cfttabnumwidth}{\mylen}


%\newlength{\mylenf}
%\settowidth{\mylenf}{\cftfigpresnum}
%\setlength{\cftfignumwidth}{\dimexpr\mylenf+1.5em}
%\setlength{\cfttabnumwidth}{\dimexpr\mylenf+1.5em}

\begin{document}

\renewcommand{\chapterpagestyle}{scrheadings}



\begin{titlepage}
\begin{center}
\large \bfseries UNIVERSIDADE FEDERAL DE SÃO PAULO

ESCOLA DE FILOSOFIA, LETRAS E CIÊNCIAS HUMANAS
\end{center}

\vspace{3cm}

\centering

\LARGE \bfseries \sffamily

NOME DO ALUNO
\vspace{5cm}


TÍTULO : Subtítulo

\vfill

\large Guarulhos \par
ANO

\end{titlepage}

%folha de rosto
\begin{titlepage}
\begin{center}
\large \bfseries UNIVERSIDADE FEDERAL DE SÃO PAULO

ESCOLA DE FILOSOFIA, LETRAS E CIÊNCIAS HUMANAS
\end{center}

\vspace{3cm}

\begin{center}




\LARGE \bfseries \sffamily 


NOME DO ALUNO
\vspace{3cm}

TÍTULO: Subtítulo
\end{center}

\bigskip


\begin{flushright}
\begin{minipage}{.5\textwidth}
\singlespacing
Trabalho de Conclusão de Curso apresentado à Universidade Federal de São Paulo como requisito parcial para obtenção do grau em Bacharel (ou Licenciado) em (nome do curso).\par 
Orientador: Nome do Orientador.
\end{minipage}
\end{flushright}

\vfill
\centering
Guarulhos \par 
Ano
\end{titlepage}

%para começar a folha de aprovacao na página 3, se nao usar os comandos para capa e folha de rosto:

\pagestyle{scrheadings}
\clearscrheadfoot
\setcounter{page}{3}



\chapter*{}
\phantom{}
\vspace{10cm}
\begin{center}
\textit{Dedicado a todos que me ajudaram nesta caminhada.}
\end{center}
\vfill
%\cleardoublepage



\tableofcontents
%\cleardoublepage

\listoffigures
%\cleardoublepage

\listoftables


\chapter*{\begin{center}Resumo\end{center}}


\chapter*{\begin{center} Abstract \end{center}}

\onehalfspacing



\addchap{Introdução}
\ohead{\normalfont \thepage}
\lipsum[1-8]

\chapter{Teste de capítulo}
\lipsum[1-8]

\section{Jean-Baptiste Camille Corot (1796–1875)}

\begin{figure}[htpb]
\centering
\includegraphics[width=.8\textwidth]{avray}
\caption[Jean-Baptiste Camille Corot. \textit{Ville d'Avray}. (ca. 1867).]{Jean-Baptiste Camille Corot. \textit{Ville d'Avray}. (ca. 1867). Óleo sobre tela. National Gallery, Washington DC. Fonte: Wikipedia (\url{http://fr.wikipedia.org/wiki/Fichier:Corot.villedavray.750pix.jpg})}
\label{corot}
\end{figure}



\chapter{Novo Capítulo}

Este é um capítulo com exemplo da nova regra da ABNT para figuras. 
Deve-se usar o pacote \texttt{caption}: \verb+\usepackage{caption}+.

\begin{verbatim}
\begin{figure}[htbp]
  \centering
  \caption{Figura de teste}
  \rule{4cm}{3cm}%para simular uma figura real
    \label{tab:test}
  %\captionsetup{font={footnotesize,bf,it}}
  \caption*{A fonte da figura}
\end{figure}
\end{verbatim}

\begin{figure}[htbp]
  \centering
  \caption{Figura de teste}
  \rule{4cm}{3cm}%para simular uma figura real
    \label{tab:test}
  %\captionsetup{font={footnotesize,bf,it}}
  \caption*{A fonte da figura}
\end{figure}


\begin{table}[htdp]
\caption{Exemplo de Tabela}
\begin{center}
\begin{tabular}{|c|c|c|}
A & B & C\\
\hline
D & E & F\\
\end{tabular}
\end{center}
\label{default}
\end{table}

\addchap{Apêndice A}


\end{document}
