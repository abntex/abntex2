\documentclass[a4paper]{abnt}
\usepackage[utf8]{inputenc}
\usepackage[dvipsnames]{xcolor}

\begin{document}
  \instituicao{INSTITUTO EDUCACIONAL SANTOS ELIAS - SOBRADINHO DF}
       \autor{Pedro de Alcântara João Carlos Leopoldo Salvador Bibiano Francisco Xavier de Paula Leocádio Miguel Gabriel Rafael Gonzaga}
        \orientador{Pedro de Alcântara João Carlos Leopoldo Salvador Bibiano Francisco Xavier de Paula Leocádio Miguel Gabriel Rafael Gonzaga Junior}
        \coorientador{João Carlos Leopoldo Salvador\\ Pedro  Salvador \\ Marcos Brasil }
        \titulo{Is Egypt's new president a reformer?}
        \comentario{Estrangeiros sem residência permanente conseguem o documento porque, no processo de inscrição, o formulário pede como identificação os números do Cadastro de Pessoa Física (CPF), que pode ser obtido por qualquer pessoa, e do Registro Nacional de Estrangeiro (RNE), que estrangeiros possuem mesmo sem ter residência fixa no país. Depois, é necessário informar o número do recibo da declaração do Imposto de Renda ou, na ausência desse, do título de eleitor. Mas qualquer pessoa pode fazer uma declaração de isento e obter o número de recibo do IR.}
        \local{Rio de Janeiro}
        \data{22/12/2012}
\folhaderosto

%~~~~~~~~~~~~~~~~~~~~~~~~~~~~~~~~~~~~~~~~~~~~~~~~~~~~~~~~~~~~~~~~~~~~~
%
%    File      : abstract
%    Type      : TeX
%    Date      : terça-feira, março 19, 2012 at 10:04
%
%    Content   : Coloque aqui o resumo de sua tese/dissertação em língua estrangeira
%~~~~~~~~~~~~~~~~~~~~~~~~~~~~~~~~~~~~~~~~~~~~~~~~~~~~~~~~~~~~~~~~~~~~~

%~~~~~~~~~~~~~~~~~~~~~~~~~~~~~~~~~~~~~~~~~~~~~~~~~~~~~~~~~~~~~~~~~~~~~
%    Obrigatório, pela ABNT.
%    Versão em língua estrangeira do resumo. Obrigatório, pela ABNT. O
%    título é ABSTRACT, em inglês, RESUMEN, em espanhol castelhano, e
%    RÉSUMÉ, em francês.
%~~~~~~~~~~~~~~~~~~~~~~~~~~~~~~~~~~~~~~~~~~~~~~~~~~~~~~~~~~~~~~~~~~~~~


\begin{abstract}

The present work, conducted within the School of Brasilia,\ldots

\textbf{Key-words:} 
%TODO NO PROJETO PGTEL PALAVRAS CHAVES FOI DEFINIDA%
%	\palavrasChavesIngles
\end{abstract}


%~~~~~~~~~~~~~~~~~~~~~~~~~~~~~~~~~~~~~~~~~~~~~~~~~~~~~~~~~~~~~~~~~~~~~
%
%    File      : agradecimento
%    Type      : TeX
%    Date      : terça-feira, março 19, 2012 at 09:33
%
%    Content   : Coloque aqui seus agradecimentos.
%~~~~~~~~~~~~~~~~~~~~~~~~~~~~~~~~~~~~~~~~~~~~~~~~~~~~~~~~~~~~~~~~~~~~~

\chapter*{Agradecimentos}

Agradeço especialmente\ldots

Por fim, meu maior e mais singelo agradecimento é direcionado à Deus, por
permitir que esta incrível aventura seja possível todos os dias.

\mbox{}
\newpage


%~~~~~~~~~~~~~~~~~~~~~~~~~~~~~~~~~~~~~~~~~~~~~~~~~~~~~~~~~~~~~~~~~~~~~
%
%    File      : capa
%    Type      : TeX
%    Date      : terça-feira, março 01, 2011 at 09:42
%
%    Content   : Informações para gerar a capa e folha de rosto automaticamente
%~~~~~~~~~~~~~~~~~~~~~~~~~~~~~~~~~~~~~~~~~~~~~~~~~~~~~~~~~~~~~~~~~~~~~

%TODO FORAM REMOVIDOS DA CAPA ORIENTADOR , INSTITUIÇÃO EMAIL E COMENTÁRIO
\autor{GUILHERME LOPES DOS SANTOS FONSECA}
\titulo{Fisolofia Abstrata em Atitudes Concretas}
\orientador[Orientador:]{Alvaros Marques Azevedo}

\instituicao{
   \footnotesize{Universidade de Brasília -- UnB            \par
                      Faculdade de Ciência da Informação   	\par
                      Programa de Pós-Graduação   				\par
                      \texttt{guilhermelsfonseca@gmail.com}%
                }}
%TODO CAPA NÃO TEM COMENTÁRIO MAIS%
\comentario{Uma ilusão do cotidiano lúdico}
\local{Brasília}
\data{07 de maio 1998}
\capa

%CONTEUDO DA DECIDATORIA NAO EH ABNTEX%%%

\newpage
\addcontentsline{toc}{chapter}{Dedicatória}

\begin{flushright}

\begin{minipage}[b]{8.9cm}
\vspace{17.01cm}
Esta obra é dedicada às coisas: àquelas ínfimas e minúsculas das quais são
feitos os mundos, as estrelas, as galáxias e os universos; àquelas imensas e
colossais que moldam a beleza, consomem a morte, alimentam os sonhos,
fertilizam as ideias e iluminam o amor; às coisas que são e às que não são; às
que sentem e às que doem; às que sorriem e às que queimam; às que choram e às
divinas; às que deliram e às que dormem; às terrenas e às de Vênus\ldots~às que
existem e às que nunca existiram.\\
(Lauro César Araujo)
\end{minipage}

\end{flushright}




%*********************************************************************
%   EPÍGRAFE TB NÃO É UM ABNTEX SIMPLIFICADO
%*********************************************************************

\cleardoublepage

{\color{white}
 \chapter*{Epígrafe}
}

\begin{flushright}
\vspace{17cm}
\emph{'Não vos amoldeis às estruturas deste mundo, \\
mas transformai-vos pela renovação da mente, \\
a fim de distinguir qual é a vontade de Deus: \\
o que é bom, o que Lhe é agradável, o que é perfeito.\\
(Bíblia Sagrada, Romanos 12, 2)}
\end{flushright}

\end{document}


