%
% Documento de teste do abnTeX2
%
\documentclass[12pt,openright,twoside,a4paper]{abntex2}
%\documentclass[12pt,oneside,a4paper]{abntex2}

% ----------------------------------------------------------
% PACOTES
% ----------------------------------------------------------

% ---
% Pacotes fundamentais 
% ---
\usepackage[T1]{fontenc}		% seleção de códigos de fonte.
\usepackage[utf8]{inputenc}		% determina a codificação utiizada (conversão automática dos acentos)
\usepackage{makeidx}            % cria o indice
\usepackage{hyperref}  			% controla a formação do índice
\usepackage{cmap}				% Mapear caracteres especiais no PDF
\usepackage{lastpage}			% usado pela fichacatalografica.tex
\usepackage{indentfirst}		% Identa o primeiro parágrafo de cada seção.
\usepackage{nomencl} 			% Lista de simbolos
% \usepackage[alf]{abntcite}					% Citações padrão ABNT

  
% ---
% PACOTES ADICIONAIS, APENAS PARA TESTE DE FUNCIONALIDADES
% ---
\usepackage{lipsum}				% para geração de dummy text
\usepackage{mychemistry}	    % Desenho de estruturas químicas
    
% ---
% Informações de dados para CAPA e FOLHA DE ROSTO
% ---
\titulo{Modelo Canônico de Dissertação, Tese e monografias em geral}
\autor{Equipe do abn\TeX2}
\local{Brasília, DF}
\data{2012}
\orientador{Lauro César Araujo}
\coorientador{Mamede Lima-Marques}
\instituicao{%
  Universidade de Brasília -- UnB
  \par
  Faculdade de Ciência da Informação
  \par
  Programa de Pós-Graduação}
\tipotrabalho{Tese (Doutorado)}
% O preambulo deve conter o tipo do trabalho, o objetivo, o nome da instituição
% e a área de concentração 
\preambulo{Tese de doutorado apresentada à Faculdade de Ciência da Informação
da Universidade de Brasília como requisito parcial para obtenção do título de
Doutor em Ciência da Informação.}
% ---

% ---
% CONFIGURACOES DE PACOTES
% ---
\hypersetup{
     	backref=true,
     	%pagebackref=true,
		%bookmarks=true,         		% show bookmarks bar?
		pdftitle={\imprimirtitulo}, 
		pdfauthor={\imprimirautor},
		pdfkeywords={PALAVRAS CHAVES EM PORTUGUES},
    		pdfsubject={\imprimirpreambulo},
	    pdfproducer={LaTeX with abnTeX2}, 	% producer of the document
    		colorlinks=true,       		% false: boxed links; true: colored links
    		linkcolor=blue,          	% color of internal links
    		citecolor=blue,        		% color of links to bibliography
    		filecolor=magenta,      		% color of file links
		urlcolor=blue,
		bookmarksdepth=4
}
% --- 



% ---
% compila o indice
% ---
\makeindex
% ---

% ---
% Compila a lista de abreviaturas e siglas
% ---
\makenomenclature
% ---

% ----
% Início do documento
% ----
\begin{document}

% ----------------------------------------------------------
% ELEMENTOS PRÉ-TEXTUAIS
% ----------------------------------------------------------
% \pretextual

% ---
% Capa
% ---
\imprimircapa
% ---

% ---
% Folha de rosto
% ---
\imprimirfolhaderosto*
% ---

% ---
% Inserir a ficha bibliografica
% ---
\thispagestyle{empty}

%
% Isto é um exemplo de Ficha Catalográfica, ou ``Dados internacionais de
% catalogação-na-publicação''. Você pode utilizar este modelo temporariamente.
% Porém, provavelmente a biblioteca da sua universidade lhe fornecerá um PDF
% com a ficha definitiva. Quando estiver com este documento, substitua todo o
% conteúdo de implementação pelo simples include do PDF abaixo:
%

% \includepdf{fig_ficha_catalografica.pdf}


{
\vspace*{15cm}					% Posição vertical
\footnotesize					% Tamanho da letra (igual às notas de rodapé)
\hrule							% Linha horizontal
\begin{center}					% Minipage Centralizado
\begin{minipage}[c]{12.5cm}		% Largura
\begin{spacing}{1.0}				% Espaçamento da linha

\imprimirautor

\hspace{0.5cm} \imprimirtitulo  / \imprimirautor. --
\imprimirlocal, imprimirdata-

\hspace{0.5cm} \pageref{LastPage} p. : il. (algumas color.) ; 30 cm.
\\

\hspace{0.5cm} Orientador: \imprimirorientadorConteudo
\\

\hspace{0.5cm} \imprimirtipotrabalho~--~\imprimirinstituicao, \imprimirdata.
\\

\hspace{0.5cm} Bibliografia: p. 253--273.
\\

\hspace{0.5cm}
	1. Palavra-chave1.
	2. Palavra-chave2.
	I. Orientador.
	II. Universidade xxx.
	III. Faculdade de xxx.
	IV. Título 			% é assim mesmo
\\

\hspace{8.75cm} CDU 02:141:005.7
\\

\end{spacing}
\end{minipage}
\end{center}
\hrule
}
% ---

% ---
% Inserir folha de aprovação
% ---
%
% Isto é um exemplo de Folha de aprovação, elemento obrigatório da NBR
% 14724/2011 (seção 4.2.1.3). Você pode utilizar este modelo até que a aprovação
% do trabalho. Após isso, substitua todo o conteúdo deste arquivo por uma
% imagem da página assinada pela banca com o comando abaixo:
%
% \includepdf{folhaaprovacao_final.pdf}
% \cleardoublepase

\begin{folhadeaprovacao}
  \clearpage
  \begin{center}
    \vspace*{1cm}
    {\ABNTEXchapterfont\large\imprimirautor}

    \vspace*{\fill}\vspace*{\fill}
    {\ABNTEXchapterfont\Large\imprimirtitulo}
    \vspace*{\fill}
    
    \hspace{.45\textwidth}
    \begin{minipage}{.5\textwidth}
        \imprimirpreambulo
    \end{minipage}%
    \vspace*{\fill}
   \end{center}
    
   Trabalho aprovado. \imprimirlocal, 24 de novembro de 2012:

   \assinatura{\textbf{\imprimirorientador} \\ Orientador} 
   \assinatura{\textbf{Professor} \\ Convidado 1}
   \assinatura{\textbf{Professor} \\ Convidado 2}
%    \assinatura{\textbf{Professor} \\ Convidado 3}
%    \assinatura{\textbf{Professor} \\ Convidado 4}
      
   \begin{center}
    \vspace*{0.5cm}
    {\large\imprimirlocal}
    \par
    {\large\imprimirdata}
    \vspace*{1cm}
  \end{center}
  
\end{folhadeaprovacao}
% ---

% ---
% Dedicatória
% ---
\begin{dedicatoria}
Dedico este trabalho ao infinito.
\end{dedicatoria}
% ---

% ---
% Agradecimentos
% ---
\begin{agradecimentos}
Os agradecimentos vão aqui.
\end{agradecimentos}
% ---

% ---
% Epígrafe
% ---
\begin{epigrafe}
    \vspace*{\fill}
	\begin{flushright}
		\textit{``Não vos amoldeis às estruturas deste mundo, \\
		mas transformai-vos pela renovação da mente, \\
		a fim de distinguir qual é a vontade de Deus: \\
		o que é bom, o que Lhe é agradável, o que é perfeito.\\
		(Bíblia Sagrada, Romanos 12, 2)}
	\end{flushright}
\end{epigrafe}
% ---

% ---
% RESUMOS
% ---

% Resumo em português  {título do resumo}{palavras chaves}
\begin{resumo}
 Segundo a ABNT NBR 6028/2003:
 3.1 O resumo deve ressaltar o objetivo, o método, os resultados e as
 conclusões do documento. A ordem e a extensão destes itens dependem do tipo
 de resumo (informativo ou indicativo) e do tratamento que cada item recebe no
 documento original. 3.2 O resumo deve ser precedido da referência do
 documento, com exceção do resumo inserido no próprio documento.
    
 \vspace{\onelineskip}
  
 \noindent
 \textbf{Palavras-chaves}: latex. abntex. editoração de texto.
\end{resumo}

% Resumo em inglês
\begin{resumo}[Abstract]
 This is the english abstract.

 \vspace{\onelineskip}
 
 \noindent 
 \textbf{Key-words}: latex. abntex. text editoration.
\end{resumo}

% Resumo em francês 
\begin{resumo}[Résumé]
  Ceci est le résumé français.
 
 \vspace{\onelineskip}
 
 \noindent
 \textbf{Mots-clés}: latex. abntex. text editoration.
\end{resumo}

% Resumo em espanhol
\begin{resumo}[Resumen]
  Este es el resumen de español.
  
 \vspace{\onelineskip}
 
 \noindent
 \textbf{Palabras clave}: latex. abntex. text editoration.
\end{resumo}

% ---
% inserir lista de ilustrações
% ---
\pdfbookmark[0]{\listfigurename}{lof}
\listoffigures*
\cleardoublepage
% ---

% ---
% inserir lista de tabelas
% ---
\pdfbookmark[0]{\listtablename}{lot}
\listoftables*
\cleardoublepage
% ---

% ---
% inserir lista de Abreviaturas e Siglas
% ---
\nomenclature{Fig.}{Figure}
\nomenclature{$A_i$}{Area of the $i^{th}$ component} 
\nomenclature{456}{Isto é um número}
\nomenclature{4456}{Isto é outro número}
\nomenclature{a}{primeira letra do alfabeto}
\nomenclature{lauro}{meu nome} 

\renewcommand{\nomname}{Lista de Abreviaturas e Siglas}
\pdfbookmark[0]{\nomname}{las}
\printnomenclature
\cleardoublepage
% ---

% ---
% inserir lista de símbolos
% ---
% ---

% ---
% inserir o sumario
% ---
\pdfbookmark[0]{\contentsname}{toc}
\tableofcontents*
\cleardoublepage
% ---



% ----------------------------------------------------------
% ELEMENTOS TEXTUAIS
% ----------------------------------------------------------
\textual

% ----------------------------------------------------------
% Introdução
% ----------------------------------------------------------
\addcontentsline{}{}{}
\chapter*{Introdução}
\addcontentsline{toc}{chapter}{Introdução}


\lipsum[1-2]

% ----------------------------------------------------------
% Parte de preparação da pesquisa
% ----------------------------------------------------------
\part{Preparação da pesquisa}

% ----------------------------------------------------------
% Capitulo 1
% ----------------------------------------------------------
\chapter{Lorem ipsum dolor sit amet}

\section{Este é um nome muito longo de uma seção sem nenhuma outra qualificação
adicional que não seja a própria qualidade ter um nome longo}

\lipsum[3]

Alineas:

\begin{alineas}
  \item Item 1\footnote{As notas devem ser digitadas ou datilografadas
  dentro das margens, ficando separadas do texto por um espaço simples de entre as
  linhas e por filete de 5 cm, a partir da margem esquerda. Devem ser
  alinhadas, a partir da segunda linha da mesma nota, abaixo da primeira letra
  da primeira palavra, de forma a destacar o expoente, sem espaço entre elas e
  com fonte menor. NBR 14724/2011 - 5.2.1}
  \item \lipsum[41]
  \item Item 3\footnote{Item 3 footnote.}:
  \begin{alineas}
    \item \lipsum[50]
    \item \lipsum[51]
  \end{alineas}
  \begin{incisos}
    \item \lipsum[52]
    \item \lipsum[53]
  \end{incisos}
  \item \lipsum[60]
\end{alineas}


\lipsum[4]

\begin{table}[htb]
\footnotesize
\caption[Níveis de investigação]{\footnotesize{Níveis de investigação.
\cite{van86}}}
\label{tab-nivinv}
\begin{tabular}{p{2.6cm}|p{6.0cm}|p{2.25cm}|p{3.40cm}}
  %\hline
   \textbf{Nível de Investigação} & \textbf{Insumos}  & \textbf{Sistemas de Investigação}  & \textbf{Produtos}  \\
    \hline
    Meta-nível & Filosofia\index{Filosofia} da Ciência  & Epistemologia &
    Paradigma  \\
    \hline
    Nível do objeto & Paradigmas do metanível e evidências do nível inferior &
    Ciência  & Teorias e modelos \\
    \hline
    Nível inferior & Modelos e métodos do nível do objeto e problemas do nível inferior & Prática & Solução de problemas  \\
   % \hline
\end{tabular}
\end{table}

\lipsum[5]

% figura química 

\begin{figure}[htb]
	\caption{\label{fig_aromaticos}Configurações de moléculas químicas}


	\begin{center}
		\makevisible
		\colorlet{mCgreen}{green!50!gray}
		\colorlet{mCblue}{cyan!50!gray}
		\colorlet{mCred}{magenta!50!gray}
		\colorlet{mCyellow}{yellow!50!gray}
		\begin{rxn}
			\tikzset{reactant/.style={draw=#1,fill=#1!10,inner sep=1em,minimum height=10em,minimum width=12em,rounded corners}}
			\reactant[,cytosine,reactant=mCred]{\chemfig{H-[:30]N*6(-(=O)-N=(-NH _2)-=-)}}
			\anywhere{cytosine.-90,,yshift=-2mm}{Citosina}
			\reactant[,thymine,reactant=mCyellow]{\chemfig{H-[:30]N*6(-(=O)-N(-H)
			-(=O)-(-CH_3)=-)}}
			\anywhere{thymine.-90,,yshift=-2mm}{Timina}
			\reactant[cytosine.-90,adenine,yshift=-2em,reactant=mCblue]{\chemfig
			{[:-36]*5(-N(-H)-*6(-N=-N=(-NH_2)--)--N=)}}
			\anywhere{adenine.-90,Guanin,yshift=-2mm}{Adenina}
			\reactant[,guanine,reactant=mCgreen]{\chemfig{[:-36]*5(-N(-H)-*6(-N=(-
			NH_2)-N(-H)-(=O)--)--N=)}}
			\anywhere{guanine.-90,,yshift=-2mm}{Guanina}
		\end{rxn}
	
	\end{center}
	
	\legend{Fonte: os autores}
	
\end{figure}

\lipsum[6]


\section{Seção 1}

\lipsum[7]

\begin{directcite}
As citações diretas, no texto, com mais de três linhas, devem ser
destacadas com recuo de 4 cm da margem esquerda, com letra menor que a do texto
utilizado e sem as aspas. No caso de documentos datilografados, deve-se
observar apenas o recuo.
\end{directcite}

Você também pode utilizar o comando ``citacao'', que é o mesmo que ``directcite'':

\begin{citacao} (ABNT NBR 10520/2002 - 5.3) 5.1
Formato (...)
Recomenda-se, quando digitado, a fonte tamanho 12 para todo o trabalho,
inclusive capa, excetuando-se citaçõess com mais de três linhas, notas de
rodapé, paginação, dados internacionais de catalogação-na-publicação,
legendas e fontes das ilustrações e das tabelas, que devem ser em tamanho menor
e uniforme.
\end{citacao}

\subsection{Subseção 1}

\lipsum[8-10]

\subsection{Subseção 2}

Plain text.

\lipsum[11-15]

\subsection{Subseção 3}

More plain text.

\lipsum[16-20]

More plain text\index{plain text}.

\subsubsection{Subsubseção 3.1}

More plain text\index{plain text}.

\lipsum[21-25]


% ----------------------------------------------------------
% Parte de revisãod e literatura
% ----------------------------------------------------------
\part{Revisão de Literatura}

% ---
% capitulo Segundo assunto
% ---

\chapter{Etiam eget ligula eu lectus lobortis condimentum}

\lipsum[1-1]

\begin{figure}[htb]
	\caption{\label{fig_circulo}A delimitação do espaço}
	\begin{center}
	    \setlength{\unitlength}{5cm}
		\begin{picture}(1,1)
		\put(0,0){\line(0,1){1}}
		\put(0,0){\line(1,0){1}}
		\put(0,0){\line(1,1){1}}
		\put(0,0){\line(1,2){.5}}
		\put(0,0){\line(1,3){.3333}}
		\put(0,0){\line(1,4){.25}}
		\put(0,0){\line(1,5){.2}}
		\put(0,0){\line(1,6){.1667}}
		\put(0,0){\line(2,1){1}}
		\put(0,0){\line(2,3){.6667}}
		\put(0,0){\line(2,5){.4}}
		\put(0,0){\line(3,1){1}}
		\put(0,0){\line(3,2){1}}
		\put(0,0){\line(3,4){.75}}
		\put(0,0){\line(3,5){.6}}
		\put(0,0){\line(4,1){1}}
		\put(0,0){\line(4,3){1}}
		\put(0,0){\line(4,5){.8}}
		\put(0,0){\line(5,1){1}}
		\put(0,0){\line(5,2){1}}
		\put(0,0){\line(5,3){1}}
		\put(0,0){\line(5,4){1}}
		\put(0,0){\line(5,6){.8333}}
		\put(0,0){\line(6,1){1}}
		\put(0,0){\line(6,5){1}}
		\end{picture}
	\end{center}
	\legend{Fonte: os autores}
	
\end{figure}

\lipsum[2-3]

\section{Seção 1}

\lipsum[4-5]

\subsection{Subseção 1}

Plain text.

\lipsum[6-10]

\subsection{Subseção 2}

Plain text\index{plain text}.

\lipsum[11-15]

\subsection{Subseção 3}

More plain text.

\lipsum[16-17]

\subsubsection{Subsubseção 3.1}

More plain text.

\lipsum[18-20]

% ----------------------------------------------------------
% Resultados
% ----------------------------------------------------------
\part{Resultados}

% ---
% capitulo de Resultados
% ---
\chapter{Lectus lobortis condimentum}

\lipsum[21-23]

\chapter{Nam sed tellus sit amet lectus urna ullamcorper tristique interdum
elementum}

\lipsum[24]

% ---
% Finaliza a parte no bookmark do PDF
% ---
\bookmarksetup{startatroot}% this is it
% ---

% ----------------------------------------------------------
% ELEMENTOS PÓS-TEXTUAIS
% ----------------------------------------------------------
\postextual

% ----------------------------------------------------------
% Conclusão
% ----------------------------------------------------------

\chapter*{Conclusão}
\addcontentsline{toc}{chapter}{Conclusão}

\lipsum[31-33]


% ----------------------------------------------------------
% Referências bibliográficas
% ----------------------------------------------------------
\bibliographystyle{plain}	% (uses file "plain.bst")
\bibliography{references}


% ----------------------------------------------------------
% Glossário
% ----------------------------------------------------------
%
% Há diversas soluções prontas para glossário. Não é necessário nos preocuparmos
% com isso agora.
%
%\glossary

% ----------------------------------------------------------
% Apêndices
% ----------------------------------------------------------

% ---
% Inicia os apêndices
% ---
\begin{apendicesenv}
% ---

% Imprime uma página indicando o início dos apêndices
\appendixpage

% ----------------------------------------------------------
\chapter{Primeiro apêndice}
% ----------------------------------------------------------

\lipsum[50-52]

% ----------------------------------------------------------
\chapter{Segundo apêndice com um nome bem logo de modo}
% ----------------------------------------------------------
\lipsum[55-57]
\end{apendicesenv}

% ----------------------------------------------------------
% Anexos
% ----------------------------------------------------------

% ---
% Inicia os anexos
% ---
%\anexos
\begin{anexosenv}
% 

% Imprime uma página indicando o início dos anexos
\appendixpage

% ---
\chapter{Primeiro anexo}
% ---
\lipsum[200-202]

% ---
\chapter{Segundo anexo}
% ---

\lipsum[210-212]

% ---
\chapter{Terceiro anexo}
% ---

\lipsum[213-214]

\end{anexosenv}


%---------------------------------------------------------------------
% INDICE REMISSIVO
%---------------------------------------------------------------------

% \cleardoublepage
% \phantomsection 
\printindex

\end{document}
