%% LyX 2.0.6 created this file.  For more info, see http://www.lyx.org/.
%% Do not edit unless you really know what you are doing.
\documentclass[oneside,brazil,12pt,openright,twoside,a4paper,english,french,spanish,brazil]{abntex2}
\usepackage[T1]{fontenc}
\usepackage[utf8]{inputenc}
\setcounter{secnumdepth}{3}
\setcounter{tocdepth}{3}
\usepackage{array}
\usepackage{longtable}
\usepackage{float}
\usepackage{textcomp}
\usepackage{url}
\usepackage{graphicx}
\PassOptionsToPackage{normalem}{ulem}
\usepackage{ulem}

\makeatletter

%%%%%%%%%%%%%%%%%%%%%%%%%%%%%% LyX specific LaTeX commands.
\providecommand{\LyX}{L\kern-.1667em\lower.25em\hbox{Y}\kern-.125emX\@}
%% Because html converters don't know tabularnewline
\providecommand{\tabularnewline}{\\}
%% A simple dot to overcome graphicx limitations
\newcommand{\lyxdot}{.}


%%%%%%%%%%%%%%%%%%%%%%%%%%%%%% User specified LaTeX commands.
% ---
% PACOTES
% ---
\usepackage{lastpage}			% Usado pela Ficha catalográfica
\usepackage{indentfirst}		% Indenta o primeiro parágrafo de cada seção.
\usepackage{color}				% Controle das cores
\usepackage[figurewithin=none]{caption}
% ---
% Pacotes adicionais, usados apenas no âmbito do Modelo Canônico do abnteX2
% ---
\usepackage{lipsum}				% para geração de dummy text
% ---

% ---
% Pacotes de citações
% ---
\usepackage[brazilian,hyperpageref]{backref}	 % Paginas com as citações na bibl
\usepackage[alf]{abntex2cite}	% Citações padrão ABNT

% --- 
% CONFIGURAÇÕES DE PACOTES
% --- 

% ---
% Configurações do pacote backref
% Usado sem a opção hyperpageref de backref
\renewcommand{\backrefpagesname}{Citado na(s) página(s):~}
% Texto padrão antes do número das páginas
\renewcommand{\backref}{}
% Define os textos da citação
\renewcommand*{\backrefalt}[4]{
\ifcase #1 %
Nenhuma citação no texto.%
\or
Citado na página #2.%
\else
Citado #1 vezes nas páginas #2.%
\fi}%
% ---

% --- 
% Espaçamentos entre linhas e parágrafos 
% --- 

% O tamanho do parágrafo é dado por:
\setlength{\parindent}{1.3cm}

% Controle do espaçamento entre um parágrafo e outro:
\setlength{\parskip}{0.2cm}  % tente também \onelineskip

% ---
% compila o indice
% ---
\makeindex
% ---

\makeatother

\usepackage{babel}
\begin{document}

\chapter{Resultados de comandos}


\chapterprecis{Isto é uma sinopse de capítulo. A ABNT não traz nenhuma normatização
a respeito desse tipo de resumo, que é mais comum em romances e livros
técnicos.}


\section{Codificação dos arquivos: UTF8}

A codificação de todos os arquivos do abn\TeX{}2 é \texttt{UTF8}.
É necessário que você utilize a mesma codificação nos documentos que
escrever, inclusive nos arquivos de base bibliográficas |.bib|. 


\section{Citações diretas}

Utilize o ambiente \texttt{citacao} para incluir citações diretas
com mais de três linhas:


\begin{citacao}
As citações diretas, no texto, com mais de três linhas, devem ser
destacadas com recuo de 4 cm da margem esquerda, com letra menor que
a do texto utilizado e sem as aspas. No caso de documentos datilografados,
deve-se observar apenas o recuo.
\end{citacao}
O ambiente \texttt{citacao} pode receber como parâmetro opcional um
nome de idioma previamente carregado nas opções da classe (\textbackslash{}autoref\{sec-hifenizacao\}).
No \LyX{}, esse parâmetro é ajustado através do comando ``Insert→Short
Title'', sendo que o texto da opção representa o idioma escolhido.
Nesse caso, o texto da citação é automaticamente escrito em itálico
e a hifenização é ajustada para o idioma selecionado na opção do ambiente.
Por exemplo:
\begin{citacao}[english]
Text in English language in italic with correct hyphenation.
\end{citacao}
Citações simples, com até três linhas, devem ser incluídas com aspas.
``Amor é fogo que arde sem se ver''.


\section{Notas de rodapé}

As notas de rodapé são detalhadas pela NBR 14724:2011 na seção 5.2.1%
\footnote{As notas devem ser digitadas ou datilografadas dentro das margens,
ficando separadas do texto por um espaço simples de entre as linhas
e por filete de 5 cm, a partir da margem esquerda. Devem ser alinhadas,
a partir da segunda linha da mesma nota, abaixo da primeira letra
da primeira palavra, de forma a destacar o expoente, sem espaço entre
elas e com fonte menor \textbackslash{}citeonline{[}5.2.1{]}\{NBR14724:2011\}.%
}%
\footnote{Caso uma série de notas sejam criadas sequencialmente, o abn\TeX{}
instrui o \LaTeX{} para que uma vírgula seja colocada após cada número
do expoente que indica a nota de rodapé no corpo do texto.%
}%
\footnote{Verifique se os números do expoente possuem uma vírgula para dividi-los
no corpo do texto.%
}.


\section{Tabelas}

\index{tabelas}

A \ref{tab-nvinv} é um exemplo de tabelas construídas com a interface
gráfica do \LyX{}. Comece inserindo tabelas no texto com o comando
``Insert→Float→Table'' para ter maior flexibilidade:

\begin{table}[H]
\label{tab-nvinv}\caption{Níveis de investigação}
\begin{longtable}{>{\raggedright}p{2.6cm}|>{\raggedright}p{6cm}|>{\raggedright}p{2.25cm}|>{\raggedright}p{3.4cm}}
\textbf{Nível de investigação} & \textbf{Insumos} & \textbf{Sistemas de investigação} & \textbf{Produtos}\tabularnewline
\hline 
Meta-nível & Filosofia da ciência & Epistemologia & Paradigma\tabularnewline
\hline 
Nível do objeto & Paradigmas do metanível e evidências do nível inferior & Ciência & Teorias e modelos\tabularnewline
\hline 
Nível inferior & Modelos e métodos do nível do objeto e problemas do nível inferior & Prática & Solução de problemas\tabularnewline
\end{longtable}


\legend{Fonte: \cite{van86}}
\end{table}



\section{Figuras}

Figuras podem ser criadas diretamente em \LaTeX{}, com o uso de ERT
(``Evil Red Text''), como o exemplo da \ref{fig-circulo}.

\begin{center}
\begin{figure}[H]
\label{fig-circulo}

\caption{A delimitação do espaço}


\begin{center}
	\setlength{\unitlength}{5cm} 		
	\begin{picture}(1,1) 		
		\put(0,0){\line(0,1){1}} 		
		\put(0,0){\line(1,0){1}} 		
		\put(0,0){\line(1,1){1}} 		
		\put(0,0){\line(1,2){.5}} 		
		\put(0,0){\line(1,3){.3333}} 		
		\put(0,0){\line(1,4){.25}} 		
		\put(0,0){\line(1,5){.2}} 		
		\put(0,0){\line(1,6){.1667}} 		
		\put(0,0){\line(2,1){1}} 		
		\put(0,0){\line(2,3){.6667}} 		
		\put(0,0){\line(2,5){.4}} 		
		\put(0,0){\line(3,1){1}} 		
		\put(0,0){\line(3,2){1}} 		
		\put(0,0){\line(3,4){.75}} 		
		\put(0,0){\line(3,5){.6}} 		
		\put(0,0){\line(4,1){1}} 		
		\put(0,0){\line(4,3){1}} 		
		\put(0,0){\line(4,5){.8}} 		
		\put(0,0){\line(5,1){1}} 		
		\put(0,0){\line(5,2){1}} 		
		\put(0,0){\line(5,3){1}} 		
		\put(0,0){\line(5,4){1}} 		
		\put(0,0){\line(5,6){.8333}} 		
		\put(0,0){\line(6,1){1}} 		
		\put(0,0){\line(6,5){1}} 		
	\end{picture} 	
\end{center}
\end{figure}

\par\end{center}

Ou então figuras podem ser incorporadas de arquivos externos, como
é o caso da \textbackslash{}autoref\{fig\_grafico\}. Se a figura que
ser incluída se tratar de um diagrama, um gráfico ou uma ilustração
que você mesmo produza, priorize o uso de imagens vetoriais no formato
PDF. Com isso, o tamanho do arquivo final do trabalho será menor,
e as imagens terão uma apresentação melhor, principalmente quando
impressas, uma vez que imagens vetorias são perfeitamente escaláveis
para qualquer dimensão. Nesse caso, se for utilizar o Microsoft Excel
para produzir gráficos, ou o Microsoft Word para produzir ilustrações,
exporte-os como PDF e os incorpore ao documento conforme o exemplo
abaixo. No entanto, para manter a coerência no uso de software livre
(já que você está usando \LaTeX{} e abn\TeX{}), teste a ferramenta
\textsf{InkScape}\index{InkScape} (\url{http://inkscape.org/}).
Ela é uma excelente opção de código-livre para produzir ilustrações
vetoriais, similar ao CorelDraw ou ao Adobe IllustratorAdobe Illustrator\index{Adobe Illustrator}.
De todo modo, caso não seja possível utilizar arquivos de imagens
como PDF, utilize qualquer outro formato, como JPEG, GIF, BMP, etc.
Nesse caso, você pode tentar aprimorar as imagens incorporadas com
o software livre \textsf{Gimp\index{Gimp}} (\url{http://www.gimp.org}).
Ele é uma alternativa livre ao Adobe Photoshop\index{Photoshop}.
Da mesma maneira que as tabelas, comece inserindo a figura com o comando
``Insert→Float→Figure''.

\begin{center}
\begin{figure}[H]
\caption{Gráfico produzido em excel e salvo como PDF}


\begin{centering}
\includegraphics{/home/pfessel/abntex2/doc/latex/abntex2/abntex2-modelo-img-grafico}
\par\end{centering}


\legend{Fonte: \cite[p. 24]{araujo2012}}

\end{figure}

\par\end{center}


\subsection{Figuras em \textsf{\emph{minipages}}}

\textsf{\emph{Minipages}} são usadas para inserir textos ou outros
elementos em quadros com tamanhos e posições controladas. No \LyX{}
elas são emuladas com os\linebreak{}
comandos ``Insert→Float→Figure'' e ``Insert→Float→Box''. Um roteiro
completo é dado em no Wiki do \LyX{}, em \url{http://wiki.lyx.org/Examples/FiguresSideBySide};
note que, na tela. os gráficos podem aparecer empilhados e não necessariamente
um do lado do outro. Será necessário também acertar a largura da de
cada minipage manualmente, clicando-se com o botão direito em ``Box(Minipage)''
e escolhendo ``Settings→Width'' e ``Settings→\% Page Width''.
Veja o exemplo da \textbackslash{}autoref\{fig\_minipage\_grafico1\}
e da \textbackslash{}autoref\{fig\_minipage\_grafico2\}.

\begin{figure}[H]


\begin{minipage}[t]{0.4\textwidth}%
\caption{Gráfico 1 da minipage}
\includegraphics[scale=0.2]{/home/pfessel/abntex2/doc/latex/abntex2/abntex2-modelo-img-grafico}%
\end{minipage}\hfill{}%
\begin{minipage}[t]{0.4\textwidth}%
\caption{Gráfico 2 da minipage}
\includegraphics[scale=0.2]{/home/pfessel/abntex2/doc/latex/abntex2/abntex2-modelo-img-grafico}%
\end{minipage}
\end{figure}

\begin{citacao}
Qualquer que seja o tipo de ilustração, sua identificação aparece
na parte superior, precedida da palavra designativa (desenho, esquema,
fluxograma, fotografia, gráfico, mapa, organograma, planta, quadro,
retrato, figura, imagem, entre outros), seguida de seu número de ordem
de ocorrência no texto, em algarismos arábicos, travessão e do respectivo
título. Após a ilustração, na parte inferior, indicar a fonte consultada
(elemento obrigatório, mesmo que seja produção do próprio autor),
legenda, notas e outras informações necessárias à sua compreensão
(se houver). A ilustração deve ser citada no texto e inserida o mais
próximo possível do trecho a que se refere.\cite{NBR14724:2011}
\end{citacao}

\section{Equações matemáticas}

Um dos pontos muito fortes do \LyX{} é o seu editor de equações, que
é semalhante àqueles encontrados em programas de processamento de
texto como LibreOffice e Word. Ele pode ser acessado usando-se os
comandos ``Insert→Math''.

Equações numeradas são inseridas com ``Insert→Math→Numbered~Formula'':

\begin{equation}
\forall x\in X,\;\exists y\leq\epsilon
\end{equation}


Equações matemáticas também podem ser inseridas em linha com o comando
``Insert→Math→Inline\ Formula'', como por exemplo $\lim_{x\rightarrow\infty}\exp(-x)=0$.

Para inserir expressões matemáticas sem numeração, use o comando ``Insert→Math→Display\ Formula'':

\[
\left|\sum_{i=1}^{n}a_{i}b_{i}\right|\leq\left(\sum_{i=1}^{n}a_{i}^{2}\right)\left(\sum_{i=1}^{n}b_{i}^{2}\right)
\]


Consulte mais informações sobre expressões matemáticas em \url{http://code.google.com/p/abntex2/w/edit/Referencias}.


\section{Enumerações: alíneas e subalíneas}

\index{alíneas}\index{subalíneas}\index{incisos}Quando for necessário
enumerar os diversos assuntos de uma seção que não possua título,
esta deve ser subdividida em alíneas\cite[4.3]{NBR6024:2012}:
\begin{alineas}
\item os diversos assuntos que não possuam título próprio, dentro de uma
mesma seção, devem ser subdivididos em alíneas; 
\item o texto que antecede as alíneas termina em dois pontos;
\item as alíneas devem ser indicadas alfabeticamente, em letra minúscula,
seguida de parêntese. Utilizam-se letras dobradas, quando esgotadas
as letras do alfabeto;
\item as letras indicativas das alíneas devem apresentar recuo em relação
à margem esquerda;
\item o texto da alínea deve começar por letra minúscula e terminar em ponto-e-vírgula,
exceto a última alínea que termina em ponto final;
\item o texto da alínea deve terminar em dois pontos, se houver subalínea;
\item a segunda e as seguintes linhas do texto da alínea começa sob a primeira
letra do texto da própria alínea;
\item subalíneas \cite[4.3]{NBR6024:2012} devem ser conforme as alíneas
a seguir:

\begin{alineas}
\item as subalíneas devem começar por travessão seguido de espaço;
\item as subalíneas devem apresentar recuo em relação à alínea;
\item o texto da subalínea deve começar por letra minúscula e terminar em
ponto-e-vírgula. A última subalínea deve terminar em ponto final,
se não houver alínea subsequente;
\item a segunda e as seguintes linhas do texto da subalínea começam sob
a primeira letra do texto da própria subalínea.
\end{alineas}
\item no abn\TeX{} estão disponíveis os ambientes \texttt{incisos} e \texttt{subalineas},
que em suma são o mesmo que se criar outro nível de \texttt{alineas},
como nos exemplos à seguir:

\begin{incisos}
\item \textit{Um novo inciso em itálico;}
\end{incisos}
\item Alínea em \textbf{negrito}:

\begin{subalineas}
\item \textit{Uma subalínea em itálico;}
\item \textit{\uline{Uma subalínea em itálico e sublinhado;}}
\end{subalineas}
\item Última alínea com \emph{ênfase.}\end{alineas}

\end{document}
